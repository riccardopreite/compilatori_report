\documentclass[../main]{subfiles}
\begin{document}

\section{Grammatica}
Ogni linguaggio è composto da una grammatica che definisce la struttura delle sue istruzioni, delle espressioni e dei suioi tipi.
La grammatica di \textbf{SimplanPlus}, presente in \verb|lexer/SimpLanPlus.g4|, presenta una serie di regole:
\begin{itemize}
    \item Block;
    \item Declaration;
    \item Statement;
    \item Expression;
    \item White space, comment;
    \item Errors.
\end{itemize}
Grazie a \verb|lexer/SimpLanPlus.g4| il lexer e il parser sono stati generati automaticamente. Ovviamente si è dovuto procedere all'implementazione finale
del parser per fare in modo che possa creare i giusti nodi a partire dal contesto che riceve.
\subfile{02_grammatica/02_1_blocchi}
\subfile{02_grammatica/02_2_dichiarazioni}
\subfile{02_grammatica/02_3_istruzioni}
\subfile{02_grammatica/02_4_espressioni}
\subfile{02_grammatica/02_5_commenti}
\subfile{02_grammatica/02_6_errori}

\end{document}
