\documentclass[../../main]{subfiles}
\begin{document}

\subsection{Errori semantici}
La funzione \textit{checkSemantics(Environment env)} si occupa di controllare la semantica di tutti i nodi dell'albero essendo richiamata 
ricorsivamente su tutti i figli di un node dove ognuno ritorna una lista, potenzialmente vuota, di errori semantici. Di seguito verranno illustrati
alcuni controlli di errori più significativi.

  \subsubsection{Variabili/Funzioni non dichiarate}
  Per quanto riguarda il controllo di identificatori non inizializzati basta solamente cercare se l'identificativo in questione possiede una entry
  nella symbol table dell'environment in questione, a qualsiasi livello.
  \newpage
  \begin{lstlisting}[style=java]
      
    @Override
	public ArrayList<SemanticError> checkSemantics(Environment env) {
	  
	  //create result list
	  ArrayList<SemanticError> res = new ArrayList<SemanticError>();
      entry = env.lookUp(id);
      if (entry == null)
          res.add(new SemanticError("Id "+id+" not declared"));
      else
          nestinglevel = env.getNestingLevel();

	  return res;
	}
  \end{lstlisting}
  \subsubsection{Variabili dichiarate più volte nello stesso ambiente}
  Analogalmente se si vuole controllare che una variabile sia ridichiarata più volte allo stesso livello di annidamento basta creare una entry, e se 
  il valore ritornato dall'aggiunta di quest'ultima sulla HashMap (SymbolTable) è diverso da null significa che esisteva già un valore appartente
  a quella chiave e quindi viene aggiunto un errore semantico come di seguito. Viene lasciato solo il codice per l'identificatore di una variabile in quanto quello di una funzione è speculare.
  \begin{lstlisting}[style=java]
      
    @Override
    public ArrayList<SemanticError> checkSemantics(Environment env) {
        /**
        * ...
        */
        STentry entry = new STentry(env.getNestingLevel(), type, new_offset);
        id.setEntry(entry);

        if (exp != null){
            res.addAll(exp.checkSemantics(env));
            //  dereferenceLevel = 0 because we set the status of a variable
            id.setIdStatus(new Effect(Effect.READWRITE), 0);
        }

        if ( hm.put(id.getID(),entry) != null )
            res.add(new SemanticError("Var id '"+id+"' already declared"));
        /**
        * ...
        */
    }
  \end{lstlisting}


\end{document}