\documentclass[../../main]{subfiles}
\begin{document}

\section{Espressioni}\label{s:espressioni}
Le espressioni possono essere utilizzate per assegnare valori ad una variabile, in condizioni booleane o anche come argomenti di una funzione.
\begin{lstlisting}[style=antlr]
exp	    : '(' exp ')'				                        #baseExp
    | '-' exp					                        #negExp
    | '!' exp                                           #notExp
    | lhs						                        #derExp
    | 'new' type					                    #newExp
    | left=exp op=('*' | '/')               right=exp   #binExp
    | left=exp op=('+' | '-')               right=exp   #binExp
    | left=exp op=('<' | '<=' | '>' | '>=') right=exp   #binExp
    | left=exp op=('=='| '!=')              right=exp   #binExp
    | left=exp op='&&'                      right=exp   #binExp
    | left=exp op='||'                      right=exp   #binExp
    | call                                              #callExp
    | BOOL                                              #boolExp
    | NUMBER					                        #valExp;

//Booleans
BOOL        : 'true'|'false';

//IDs
fragment CHAR 	    : 'a'..'z' |'A'..'Z' ;
ID          : CHAR (CHAR | DIGIT)* ;

//Numbers
fragment DIGIT	    : '0'..'9';
NUMBER      : DIGIT+;
\end{lstlisting}

\end{document}