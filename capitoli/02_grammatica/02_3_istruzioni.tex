\documentclass[../../main]{subfiles}
\begin{document}

\subsection{Istruzioni}\label{s:istruzioni}
Il linguaggio SimplanPlus presenta una serie di istruzioni che permettono la manipolazione di ID o chiamate di funzioni.
\begin{itemize}
    \item (\verb|assignment|) di un valore ad una variabile dichiarata;
    \item (\verb|deletion|), viene deallocata la zona di memoria alla quale il puntatore puntava;
    \item (\verb|print|), di una espressione (variabile, ritorno di funzione o int o bool);
    \item (\verb|ret|) permette ad una funzione di ritornare un valore o di ritornare al chiamante in caso di funzioni void;
    \item (\verb|ite|) definisce la costruzione di un blocco if con condizione booleana e ramo else facoltativo;
    \item (\verb|call|) invoca la funzione corrispondente con i relativi parametri attuali;
    \item (\verb|block|) rappresenta la creazione di un blocco annidato.
\end{itemize}
\begin{lstlisting}[style=antlr]
statement   : assignment ';'
        | deletion ';'
        | print ';'
        | ret ';'
        | ite
        | call ';'
        | block;

assignment  : lhs '=' exp;

lhs         : ID | lhs '^';

deletion    : 'delete' ID;

print       : 'print' exp;

ret         : 'return' (exp)?;

ite         : 'if' '(' exp ')' statement ('else' statement)?;

call        : ID '(' (exp(',' exp)*)? ')';
\end{lstlisting}



\end{document}