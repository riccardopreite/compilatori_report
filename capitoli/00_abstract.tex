\documentclass[../main]{subfiles}
\begin{document}

\begin{abstract}
Questo progetto presenta la realizzazione di un compilatore e un interprete per il linguaggio \\ \textbf{SimpLanPlus}, ideato nell'ambito del corso di Compilatori e interpreti della LM Informatica A.A. 2020/21.\\
Più nello specifico, il linguaggio adotta un paradigma imperativo, è staticamente tipato, permette la definizione di funzioni, anche ricorsive, ma non mutuamente ricorsive e la gestione di tipi puntatori.
Nelle varie fasi del progetto, descritte accuratemente nelle sezioni successive, è stata analizzata la grammatica del linguaggio e implementati svariati controlli su diversi livelli per garantire la correttezza sintattica e
semantica dei programmi, con particolare attenzione all'analisi degli effetti e al controllo dei tipi. Successivamente, è stato definito un linguaggio intermedio \textbf{SVM-Assembly} che viene preso in input da un interprete.
Quest'ultimo simula, tramite una memoria virtuale, il funzionamento di una macchina a pila in cui è permesso l'utilizzo di registri.
\end{abstract}

\end{document}