\documentclass[../../main]{subfiles}
\begin{document}

\subsection{Interprete SimplanPlus}
L'interprete di \textbf{SimpLanPlus} prende in input il codice precedentemente generato dal compilatore e lo esegue un'istruzione alla volta. 
\subsubsection{Bytecode}
Per ogni nodo dell'AST è presente un metodo \emph{String codeGeneration()} che genera il bytecode rispettando la grammatica presente nel file \emph{SVG.g4}.
Ogni istruzione è composta da: 
\begin{itemize}
    \item \textbf{Label istruzione}: è una stringa obbligatoria che identifica un'istruzone. 
    \item \textbf{Primo argomento}: serve per specificare un registro o un'etichetta. È una stringa opzionale perché non tutte le istruzioni prevedono questo argomento (es. POP).
    \item \textbf{Secondo argomento}: serve per specificare un registro o un numero. È una stringa opzionale.
    \item \textbf{Terzo argomento}: serve per specificare un registro, un numero o un'etichetta. È una stringa opzionale.
\end{itemize}
Per ogni istruzione letta viene creato un oggetto di tipo \emph{Instruction}, definito dalla classe Instruction.java.

\end{document}