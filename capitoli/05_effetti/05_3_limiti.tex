\documentclass[../../main]{subfiles}
\begin{document}

\subsection{Limiti}

Uno dei limiti del controllore di tipi implementato è l'insensibilità alla presenza di valori non dichiarati. In particolare, dal momento che la funzione di sequenzializzazione fra un valore $\bot$ ed uno $rw$ è $rw$, qualora siano presenti valori non inizializzati, il sistema di controllo degli effetti non è in gradi di rilevare l'errore.

Gli errori dovuti alla non inizializzazione dei puntatori vengono rilevati dinamicamente: ogni volta che il valore dello stack pointer viene incrementato, i valori nell'intervallo compreso fra i due valori vengono contrassegnati come illeggibili. Similmente, a seguito della liberazione di un'area dello heap tale area viene contrassegnata come illeggibile.

La macchina virtuale è in grado di sollevare un errore dinamicamente qualora venga rilevato un accesso ad un valore illeggibile sollevando un'eccezione. Dal momento che anche a seguito della contrassegnazione di non leggibilità anche a seguito di cancellazione di aree allocate, anche errori dovuti a situazioni complesse di aliasing dovute ad esempio ad assegnazioni di puntatori dereferenziati parzialmente sono certamente rilevate dinamicamente dalla virtual machine che abbiamo implementato.

 
\end{document}
